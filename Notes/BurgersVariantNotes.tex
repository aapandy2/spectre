%// Distributed under the MIT License.
%// See LICENSE.txt for details.

\documentclass[12pt]{article}
\usepackage{graphicx}
\usepackage{amsmath}
\usepackage{amsthm}
\usepackage{amssymb}
\usepackage{color}
\usepackage{braket}
\usepackage{bm}
\usepackage[margin=1in]{geometry}
\usepackage{mathtools}
\usepackage{tikz}
\usepackage{listings}

\allowdisplaybreaks
\numberwithin{equation}{section}

\interfootnotelinepenalty=10000

\usepackage{calligra}
\PassOptionsToPackage{hyphens}{url}\usepackage{hyperref}
\hypersetup{colorlinks=true, linkcolor=blue, citecolor=violet}

\DeclareMathAlphabet{\mathcalligra}{T1}{calligra}{m}{n}
\DeclareFontShape{T1}{calligra}{m}{n}{<->s*[2.2]callig15}{}
\newcommand{\scriptr}{\mathcalligra{r}\,}
\newcommand{\boldscriptr}{\pmb{\mathcalligra{r}}\,}

\newcommand{\Lagr}{\mathcal{L}}
\newcommand{\Hami}{\mathcal{H}}
\newcommand{\reals}{\rm I\!R}
\newcommand{\order}{\mathcal{O}}
\newcommand{\bx}{\mathbf{x}}
\newcommand{\bp}{\mathbf{p}}
\newcommand{\bq}{\mathbf{q}}
\newcommand{\redtext}[1]{\textcolor{red}{#1}}
\newcommand{\pvec}[1]{\vec{#1}\mkern2mu\vphantom{#1}}
\newcommand{\mA}{\mathcal{A}}

%\setcounter{section}{-1}

\begin{document}
\title{Implementing the Burgers equation variant}
\author{Alex Pandya}
\date{\today}
\maketitle
%\tableofcontents
%\clearpage

\section{Burgers equation}

A common test of shock-capturing methods is the inviscid Burgers equation,
\begin{equation} \label{eq:Burgers_PDE}
0 = \partial_t U + \partial_x \Big( \frac{1}{2} U^2 \Big)
\end{equation}
which has characteristics given by the trajectories
\begin{equation}
x(t) = U \big( t, x(t) \big)
\end{equation}
(see \cite{LeVeque92} equation 3.17 for a derivation).
One can see that if $\partial_x U < 0$ in the initial data, then
characteristics will cross at some later time and a shock will form.

Shocks are commonly handled by solving the weak formulation of Burgers
equation rather than directly solving the PDE (\ref{eq:Burgers_PDE}).
The weak formulation is obtained by multiplying by a smooth,
compactly-supported test function $\phi(t,x)$, integrating over the
domain, and integrating by parts to move derivatives from $U$ onto $\phi$:
\begin{equation} \label{eq:Burgers_weak}
0 = \int_{0}^{\infty} dt \, \int_{-\infty}^{\infty} dx \,
    \big[ U \partial_t \phi + \frac{1}{2} U^2 \partial_x \phi \big]
    + \int_{-\infty}^{\infty} dx \, \phi(x, 0) U(x, 0)
\end{equation}
which does not run into any issues when $U$ is discontinuous, as is the
case when a shock forms.

The speed at which the shock propagates is given by the Rankine-Hugoniot
condition,
\begin{equation} \label{eq:Burgers_speed}
\begin{aligned}
s_{B} &= \frac{f(U_L) - f(U_R)}{U_L - U_R} \\
&= \frac{\frac{1}{2} U_L^2 - \frac{1}{2} U_R^2}{U_L - U_R} \\
&= \frac{1}{2} (U_L + U_R) \\
\end{aligned}
\end{equation}
where we have used that $f(U) = \frac{1}{2} U^2$ for Burgers equation to obtain
the second equality.

\section{Variant of Burgers equation}

LeVeque also provides a variant of Burgers equation (\cite{LeVeque92} equation
3.41) which is obtained by multiplying (\ref{eq:Burgers_PDE}) through by $2 U$
and rearranging, which yields
\begin{equation} \label{eq:Burgers_Variant_PDE}
0 = \partial_t (U^2) + \partial_x \Big( \frac{2}{3} U^3 \Big).
\end{equation}
If we choose data such that $U \neq 0$, then solutions to the PDEs
(\ref{eq:Burgers_PDE}) and (\ref{eq:Burgers_Variant_PDE}) should be equivalent,
since we can just cancel off the factor of $2 U$.

Crucially, the weak formulation of Burgers equation, (\ref{eq:Burgers_weak})
has different \textit{discontinuous} solutions from that of the variant
equation, whose weak solutions satisfy
\begin{equation}
\begin{aligned}
0 &= \int_{0}^{\infty} dt \, \int_{-\infty}^{\infty} dx \,
    \phi \partial_t (U^2) + \int_{0}^{\infty} dt \, \int_{-\infty}^{\infty} dx
    \, \phi \partial_x \Big( \frac{2}{3} U^3 \Big) \\
&= \int_{0}^{\infty} dt \, \int_{-\infty}^{\infty} dx \,
    \Big[ U^2 \partial_t \phi + \frac{2}{3} U^3 \partial_x \phi \Big]
    + \int_{-\infty}^{\infty} dx \, U^2(0, x) \phi(0, x) \\
\end{aligned}
\end{equation}
which is not a multiple of (\ref{eq:Burgers_weak}).

A clear indication of the difference between the two sets of weak solutions
can be seen through the shock speed, which for the variant equation is:
\begin{equation} \label{eq:Burgers_variant_speed}
\begin{aligned}
s_{BV} &= \frac{f(U_L) - f(U_R)}{U_L^2 - U_R^2} \\
&= \frac{2}{3} \frac{U_L^3 - U_R^3}{U_L^2 - U_R^2} \\
\end{aligned}
\end{equation}
which differs from (\ref{eq:Burgers_speed}).

\subsection{Characteristic speeds for the variant equation}

It will be useful to begin by defining
\begin{equation}
V \equiv U^2
\end{equation}
so that we can rewrite (\ref{eq:Burgers_Variant_PDE}) as
\begin{equation} \label{eq:Burgers_variant_V}
0 = \partial_t V + \partial_x \Big( \frac{2}{3} V^{3/2} \Big).
\end{equation}

We will apply the method of characteristics to find the characteristic curves,
which are defined to be curves $x(t)$ along which the solution does not change
($\frac{d V}{d t} = 0$):
\begin{equation}
\begin{aligned}
0 &= \frac{d}{d t} V \big( t, x(t) \big)\\
&= \frac{\partial V}{\partial t} \frac{\partial t}{\partial t}
    + \frac{\partial V}{\partial x} \frac{\partial x(t)}{\partial t} \\
&= \partial_t V + x'(t) \partial_x V \\
\end{aligned}
\end{equation}
and we see that the last line can be made equal to
(\ref{eq:Burgers_variant_V}) if we have
\begin{equation}
x'(t) = V^{1/2}(t, x)
\end{equation}
which is the same characteristic speed we found for the original Burgers
equation if we undo the variable change $V = U^2$ (which makes sense,
since smooth solutions are identical between the two and should propagate
at the characteristic speed).

\clearpage

\section{Summary: Burgers' equation and the variant}

The inviscid Burgers' equation takes the form
\begin{equation} \label{eq:Burgers_PDE_summary}
0 = \partial_t U + \partial_x \Big( \frac{1}{2} U^2 \Big).
\end{equation}

If we restrict ourselves to solutions where $U > 0$, one might expect that the
variant
\begin{equation} \label{eq:Burgers_Variant_PDE_summary}
\begin{aligned}
0 &= 2 U \Big[ \partial_t U + \partial_x \Big( \frac{1}{2} U^2 \Big) \Big] \\
&= \partial_t (U^2) + \partial_x \Big( \frac{2}{3} U^3 \Big) \\
&= \partial_t (V) + \partial_x \Big( \frac{2}{3} V^{3/2} \Big) \\
\end{aligned}
\end{equation}
where $V \equiv U^2$, should have identical solutions to
(\ref{eq:Burgers_PDE_summary}).

It turns out that smooth solutions to (\ref{eq:Burgers_PDE_summary}-
\ref{eq:Burgers_Variant_PDE_summary}) agree, but weak solutions (e.g. those
with shocks) differ---one can show that shocks propagate faster for the variant.

\clearpage

\begin{thebibliography}{10}
\bibitem{LeVeque92} LeVeque, R. J.,
    \textit{Numerical Methods for Conservation Laws} (1992)
\end{thebibliography}

\end{document}
