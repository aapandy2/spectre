%// Distributed under the MIT License.
%// See LICENSE.txt for details.

\documentclass[12pt]{article}
\usepackage{graphicx}
\usepackage{amsmath}
\usepackage{amsthm}
\usepackage{amssymb}
\usepackage{color}
\usepackage{braket}
\usepackage{bm}
\usepackage[margin=1in]{geometry}
\usepackage{mathtools}
\usepackage{tikz}
\usepackage{listings}

\allowdisplaybreaks
\numberwithin{equation}{section}

\interfootnotelinepenalty=10000

\usepackage{calligra}
\PassOptionsToPackage{hyphens}{url}\usepackage{hyperref}
\hypersetup{colorlinks=true, linkcolor=blue, citecolor=violet}

\DeclareMathAlphabet{\mathcalligra}{T1}{calligra}{m}{n}
\DeclareFontShape{T1}{calligra}{m}{n}{<->s*[2.2]callig15}{}
\newcommand{\scriptr}{\mathcalligra{r}\,}
\newcommand{\boldscriptr}{\pmb{\mathcalligra{r}}\,}

\newcommand{\Lagr}{\mathcal{L}}
\newcommand{\Hami}{\mathcal{H}}
\newcommand{\reals}{\rm I\!R}
\newcommand{\order}{\mathcal{O}}
\newcommand{\bx}{\mathbf{x}}
\newcommand{\bp}{\mathbf{p}}
\newcommand{\bq}{\mathbf{q}}
\newcommand{\redtext}[1]{\textcolor{red}{#1}}
\newcommand{\pvec}[1]{\vec{#1}\mkern2mu\vphantom{#1}}
\newcommand{\mA}{\mathcal{A}}

%\setcounter{section}{-1}

\begin{document}
\title{Eckart-frame MIS theory with only bulk viscosity}
\author{Alex Pandya}
\date{\today}
\maketitle
%\tableofcontents
%\clearpage

\section{Covariant EOM and general properties}

\subsection{Covariant EOM}

The covariant EOM can be found in \cite{Bemfica19}; note that they neglect electromagnetic fields.
They are:
\begin{align}
0 &= \nabla_{a} T^{ab} \label{eq:Tab_cons_law} \\
0 &= \nabla_{a} J^{a} \label{eq:Ja_cons_law} \\
u^a \nabla_a \Pi_B &= \frac{1}{\tau_{\Pi}} \Big( [- \zeta \nabla_a u^a] - \Pi_B \Big) - \frac{\lambda}{\tau_{\Pi}} \Pi_B^2 \label{eq:Pi_cov_EOM}
\end{align}
where
\begin{align}
T^{ab} &= e u^a u^b + (p + \Pi_B) \Delta^{ab} \label{eq:Tab} \\
J^{a}  &= \rho u^a \label{eq:Ja}
\end{align}
and we have defined
\begin{equation}
\begin{aligned}
\Delta^{ab} &\equiv g^{ab} + u^a u^b. \\
\end{aligned}
\end{equation}


\section{Coordinate EOM}

We now adopt the standard ADM metric ansatz
\begin{equation} \label{eq:metric}
ds^2 = - \alpha^2 dt^2 + \gamma_{ij} (dx^i + \beta^i dt) (dx^j + \beta^j dt)
\end{equation}
and we split the four-velocity into timelike and spacelike pieces via
\begin{equation}
u^a = W (n^a + v^a),
\end{equation}
where $n^a$ is the timelike normal vector, $v^a$ is the spatial velocity, and $W \equiv (1-v_c v^c)^{-1/2}$ is the Lorentz factor.

In the case with bulk viscosity, the Faraday tensor is unchanged and the only change in the stress-energy tensor is that
\begin{equation}
\begin{aligned}
p &\to p_B \equiv p + \Pi_B \\
h &\to h_B \equiv \frac{e + p_B}{\rho} \\
p^* &\to p_B^* = p_B + \frac{b^2}{2} \\
(\rho h)^* &\to (\rho h_B)^* = \rho h_B + b^2
\end{aligned}
\end{equation}
where the latter two quantities will appear in the equations of motion. 

The equations of motion including the bulk viscous term are
\begin{equation}
\partial_t U + \partial_i F^i(U) = S(U),
\end{equation}
where
\begin{equation}
U = \sqrt{\gamma}
\begin{bmatrix}
D \\
D Y_e \\
V_B \\
S_j \\
\tau \\
B^j \\
\Phi
\end{bmatrix}
\equiv
\begin{bmatrix}
\tilde{D} \\
\tilde{Y}_e \\
\tilde{V}_B \\
\tilde{S}_j \\
\tilde{\tau} \\
\tilde{B}^j \\
\tilde{\Phi}
\end{bmatrix}
=
\sqrt{\gamma}
\begin{bmatrix}
\rho W \\
\rho W Y_e \\
\hat{\Pi}_B W \\
(\rho h_B)^* W^2 v_j - \alpha b^0 b_j \\
(\rho h_B)^* W^2 - p_B^* - (\alpha b^0)^2 - \rho W \\
B^j \\
\Phi
\end{bmatrix},
\end{equation}
where we have defined densitized versions of the conserved variables, denoted with tildes, by absorbing the factor of the root of the spatial metric's determinant.
The fluxes are
\begin{equation}
F^i(U) =
\begin{bmatrix}
\tilde{D} v^i_{tr} \\
\tilde{Y}_e v^i_{tr} \\
\tilde{V}_B v^i_{tr} \\
\tilde{S}_j v^i_{tr} + \alpha \sqrt{\gamma} p^*_B \delta^i_j - \alpha b_j \tilde{B}^i/W \\
\tilde{\tau} v^i_{tr} + \alpha \sqrt{\gamma} p^*_B v^i - \alpha^2 b^0 \tilde{B}^i/W \\
\tilde{B}^j v^i_{tr} - \alpha v^j \tilde{B}^i + \alpha \gamma^{ij} \tilde{\Phi} \\
\alpha \tilde{B}^i - \tilde{\Phi} \beta^i
\end{bmatrix},
\end{equation}
and the sources are
\begin{equation}
S(U) =
\begin{bmatrix}
0 \\
0 \\
- \alpha \sqrt{\gamma} \hat{\Pi}_B \ln(\hat{\Pi}_B) \Big[ \frac{1}{\tau_\Pi} \Big( 1 + \lambda \xi \ln(\hat{\Pi}_B) \Big) + \mathcal{W} \Big] \\
\frac{\alpha}{2} \tilde{S}^{mn} \partial_j \gamma_{mn} + \tilde{S}_k \partial_j \beta^k - (\tilde{D} + \tilde{\tau}) \partial_j \alpha \\
\alpha \tilde{S}^{mn} K_{mn} - \tilde{S}^m \partial_m \alpha \\
-\tilde{B}^m \partial_m \beta^j + \Phi \partial_k (\alpha \sqrt{\gamma} \gamma^{jk}) \\
\tilde{B}^k \partial_k \alpha - \alpha K \tilde{\Phi} - \alpha \kappa \tilde{\Phi}
\end{bmatrix},
\end{equation}
where
\begin{equation}
\begin{aligned}
\tilde{S}^{ij} &= \sqrt{\gamma} \big[ (\rho h_B)^* W^2 v^i v^j + p^*_B \gamma^{ij} - \gamma^{ik} \gamma^{jl} b_k b_l \big] \\
\xi &\equiv \frac{\zeta}{\tau_\Pi} \\
\Pi_B &\equiv \xi \ln(\hat{\Pi}_B) \\
\mathcal{W} &\equiv u^a \nabla_a \ln (\xi) \underset{\textnormal{restrict}}{\to} 0 \\
\end{aligned}
\end{equation}
where the last line implies that we restrict to the case where $\xi$ is a spacetime constant, so $\mathcal{W} = 0$.
Ultimately this is a restriction on the class of hydrodynamic frames we can consider, since $\xi \propto \tau_{\Pi}$ which we are free to choose; \textit{I think it shouldn't limit the physics we can model with these equations.}

Note that since the stress-energy tensor includes the change $p \to p_B$, the terms $S^i, S^{ij}, \tau$ are modified from the case without bulk viscosity.

\subsection{Primitive solve}

\textbf{NOTE:} for now we're restricting to the case where the magnetic field is zero, and specializing to the Kastaun hydro inversion.

The primitive solve used in {\tt SpECTRE} only needs a minor modification to work for this system.
Assume that the conserved variables $D, S_j, \tau, V_B$ are all known.
We then compute (cf. \cite{Galeazzi13}):
\begin{equation}
	q \equiv \frac{\tau}{D}, ~~~ r \equiv \frac{\sqrt{\gamma^{ij} S_i S_j}}{D}, ~~~ k \equiv \frac{r}{1 + q}.
\end{equation}
We will now define $\rho, \epsilon, \hat{\Pi}_B, v^j$ to be functions of the variable $z \equiv W v$ using the following definitions:
\begin{equation}
\begin{aligned}
W(z) &= \sqrt{1 + z^2} \\
\rho(z) &= \frac{D}{W(z)} \\
\hat{\Pi}_B(z) &= \frac{V_B}{W(z)} \\
\Pi_B(z) &= \xi \ln(\hat{\Pi}_B) \\
\epsilon(z) &= W(z) q - z r + W(z) - 1 \\
p(z) &= \mathrm{EOS}\big[ \epsilon(z), \rho(z) \big] \\
a(z) &= \frac{p(z) + \Pi_B(z)}{\rho(z) [1 + \epsilon(z)]} &&\leftarrow (\textnormal{modified by } \Pi_B) \\
h(z) &= (1 + \epsilon(z))(1 + a(z)) \\
v^j(z) &= \frac{S^j}{D W(z) h(z)}.
\end{aligned}
\end{equation}
\textit{Note that the quantity $a$ is modified due to the presence of bulk viscosity, and that change is then inherited by $h$ and $v^j$}.
All of the assignments can be made in the order above (from top to bottom) given a value of $z$.

The value of $z$ is obtained by applying a root-finding algorithm to the function (so, starting with a guess $z_0$ and then iteratively refining)
\begin{equation}
f(z) = z - \frac{r}{h(z)},
\end{equation}
which is zero when evaluated on physical solutions to the MIS-bulk system.

\begin{thebibliography}{10}
\bibitem{Deppe22} Deppe, N., et al. \url{https://arxiv.org/abs/2109.12033}
\bibitem{BaumgarteShapiro} Baumgarte, T. \& Shapiro, S., ``Numerical Relativity: Solving Einstein's Equations on the Computer''
\bibitem{RezzollaZanotti} Rezzolla, L. \& Zanotti, O., ``Relativistic Hydrodynamics''
\bibitem{Bemfica19} Bemfica, F., Disconzi, M., \& Noronha, J. \url{https://arxiv.org/abs/1901.06701}
\bibitem{Gourgoulhon07} Gourgoulhon, E. \url{https://arxiv.org/abs/gr-qc/0703035}
\bibitem{Galeazzi13} Galeazzi, F., et al. \url{https://journals.aps.org/prd/abstract/10.1103/PhysRevD.88.064009}
\bibitem{Chabanov23a} Chabanov, M. \& Rezzolla, L., \url{https://arxiv.org/abs/2307.10464}
\bibitem{Chabanov23b} Chabanov, M., \& Rezzolla, L., \url{https://arxiv.org/abs/2311.13027}
\bibitem{PMP22b} Pandya, Most, Pretorius 22b \url{https://arxiv.org/abs/2209.09265}
\end{thebibliography}

\clearpage

\appendix

\clearpage

\section{Steady-state shockwave solutions}

In this section we will consider solutions to the system with zero magnetic field in flat spacetime.
We will further assume that $\zeta/\tau_{\Pi}$ is a spacetime constant, so the bulk evolution equation takes the form of a conservation law.

Under these constraints, the equations of motion are
\begin{align}
\nabla_a T^{ab} &= 0 \\
\nabla_a J^a &= 0\\
\nabla_a (\hat{\Pi}_B u^a) &= \mathcal{S}_\Pi,
\end{align}
where
\begin{align}
T^{ab} &= e u^a u^b + (p + \Pi_B) \Delta^{ab} \\
J^a &= \rho u^a \\
\Pi_B &= \frac{\zeta}{\tau_{\Pi}} \ln(\hat{\Pi}_B) \\
\mathcal{S}_\Pi &\equiv - \frac{1}{\tau_\Pi} \hat{\Pi}_B \ln(\hat{\Pi}_B) \Big( 1 + \frac{\zeta \lambda}{\tau_\Pi} \ln(\hat{\Pi}_B) \Big)
\end{align}
For now we'll leave the EOS implicit, $p = p(e, \rho)$.

We'll now further restrict to solutions which vary only in one spatial coordinate, which we will take (WLOG) to be $x$, and we write the four velocity in terms of the Lorentz factor $W$ and the $x$-direction coordinate velocity $v$ as
\begin{equation}
u^a = (W, W v, 0, 0)^T.
\end{equation}

Under these restrictions, the particle current equation becomes
\begin{equation} \label{eq:J_eqn}
\begin{aligned}
0 &= \partial_a J^{a} \\
&= \partial_x J^x \\
\implies J^x &\equiv \rho W v = \mathrm{const.}
\end{aligned}
\end{equation}
The stress-energy conservation laws imply
\begin{equation} \label{eq:T_eqns}
\begin{aligned}
0 &= \partial_a T^{ab} \\
&= \partial_x T^{xb} \\
\implies T^{tx} &\equiv (e + p + \Pi) W^2 v = \mathrm{const.} \\
\implies T^{xx} &\equiv (e + p + \Pi) W^2 v^2 + p + \Pi = \mathrm{const.}
\end{aligned}
\end{equation}
Finally, the bulk scalar equation implies
\begin{equation} \label{eq:intermediate_Pi_ODE}
\begin{aligned}
\mathcal{S}_\Pi &= \partial_a (\hat{\Pi}_B u^a) \\
&= (\hat{\Pi}_B u^x)' \\
&= (\hat{\Pi}_B W v)' \\
&= W v \hat{\Pi}_B' + \hat{\Pi}_B v W' + \hat{\Pi}_B W  v' \\
\mathcal{S}_\Pi &= W v \hat{\Pi}_B' + \hat{\Pi}_B W^3 v' \\
\implies \hat{\Pi}_B' &= \frac{\mathcal{S}_\Pi}{W v} - \hat{\Pi}_B W^2 \frac{v'}{v}. \\
\end{aligned}
\end{equation}

We can now use (\ref{eq:J_eqn}) to solve for $\rho$
\begin{equation}
\rho = \frac{J^x}{W v}
\end{equation}
and the first result (after the first ``implies'' arrow) from (\ref{eq:T_eqns}) gives us
\begin{equation}
\begin{aligned}
e + p + \Pi &= \frac{T^{tx}}{W^2 v} \\
\end{aligned}
\end{equation}
which can be inserted into the second result from (\ref{eq:T_eqns}) to give us
\begin{equation} \label{eq:intermediate_p}
\begin{aligned}
T^{xx} &= \frac{T^{tx}}{W^2 v} W^2 v^2 + p + \Pi \\
T^{xx} &= T^{tx} v + p + \Pi \\
\implies p &= T^{xx} - T^{tx} v - \Pi. \\
\end{aligned}
\end{equation}
We can now get $e$ using this in the first result from (\ref{eq:T_eqns}):
\begin{equation}
\begin{aligned}
\frac{T^{tx}}{W^2 v} &= e + (T^{xx} - T^{tx} v - \Pi) + \Pi \\
\implies e &= \frac{T^{tx}}{v} - T^{xx}. \\
\end{aligned}
\end{equation}

At this stage we can recognize that we have $e = e(v), \rho = \rho(v)$, and through the EOS we have $p(e, \rho) = p(e(v), \rho(v)) = p(v)$; these facts imply that we should be able to write (\ref{eq:intermediate_Pi_ODE}) as an ODE for $v$.
We start by noting that since $\hat{\Pi} = \exp(\Pi \, \tau_{\Pi}/\zeta)$,
\begin{equation}
\begin{aligned}
\hat{\Pi}' &= \frac{\partial \hat{\Pi}}{\partial \Pi} \frac{\partial \Pi}{\partial x} \\
&= \frac{\tau_{\Pi}}{\zeta} e^{\Pi \, \tau_{\Pi}/\zeta} \Pi' \\
&= \frac{\tau_{\Pi}}{\zeta} \hat{\Pi} \Pi'
\end{aligned}
\end{equation}
so differentiating (\ref{eq:intermediate_p}) with respect to $x$ gives us
\begin{equation}
\begin{aligned}
- T^{tx} v' - \Pi' &= p' \\
&= \frac{\partial p}{\partial e} e' + \frac{\partial p}{\partial \rho} \rho' \\
&= \frac{\partial p}{\partial e} \Big[ - \frac{T^{tx}}{v^2} v' \Big] + \frac{\partial p}{\partial \rho} \Big[ - \frac{J^x W}{v^2} v' \Big] \\
\implies \Pi' &= \Big[ T^{tx} \Big( \frac{\partial p}{\partial e} \frac{1}{v^2} - 1 \Big) + \frac{\partial p}{\partial \rho} \frac{J^x W}{v^2} \Big] v'
\end{aligned}
\end{equation}
thus
\begin{equation}
\hat{\Pi}' = \frac{\tau_{\Pi}}{\zeta} \hat{\Pi} \Big[ T^{tx} \Big( \frac{\partial p}{\partial e} \frac{1}{v^2} - 1 \Big) + \frac{\partial p}{\partial \rho} \frac{J^x W}{v^2} \Big] v'.
\end{equation}
Inserting this result into (\ref{eq:intermediate_Pi_ODE}) gives us
\begin{equation}
\begin{aligned}
\hat{\Pi}' &= \frac{\mathcal{S}_\Pi}{W v} - \hat{\Pi} W^2 \frac{v'}{v} \\
\implies \frac{\tau_{\Pi}}{\zeta} \hat{\Pi} \Big[ T^{tx} \Big( \frac{\partial p}{\partial e} \frac{1}{v^2} - 1 \Big) + \frac{\partial p}{\partial \rho} \frac{J^x W}{v^2} \Big] v' &= \frac{\mathcal{S}_\Pi}{W v} - \hat{\Pi} W^2 \frac{v'}{v} \\
\frac{\tau_{\Pi}}{\zeta} \hat{\Pi} \Big[ T^{tx} \Big( \frac{\partial p}{\partial e} \frac{1}{v^2} - 1 \Big) + \frac{\partial p}{\partial \rho} \frac{J^x W}{v^2} \Big] v' &= \frac{- \frac{1}{\tau_\Pi} \hat{\Pi} \ln(\hat{\Pi}) \Big( 1 + \frac{\zeta \lambda}{\tau_\Pi} \ln(\hat{\Pi}) \Big)}{W v} - \hat{\Pi} W^2 \frac{v'}{v} \\
\frac{\tau_{\Pi}}{\zeta} \Big[ T^{tx} \Big( \frac{\partial p}{\partial e} \frac{1}{v^2} - 1 \Big) + \frac{\partial p}{\partial \rho} \frac{J^x W}{v^2} \Big] v' &= \frac{- \frac{1}{\tau_\Pi} \ln(\hat{\Pi}) \Big( 1 + \frac{\zeta \lambda}{\tau_\Pi} \ln(\hat{\Pi}) \Big)}{W v} - W^2 \frac{v'}{v} \\
\end{aligned}
\end{equation}
where we have cancelled a factor of $\hat{\Pi}$ from the fourth line (which is fine because $\hat{\Pi} > 0$).
We can now see that we have a couple factors of $\hat{\Pi}$ hanging around, and we'd like to express them in terms of $v$; this can be done using the EOS and (\ref{eq:intermediate_p}), via
\begin{equation}
\Pi = T^{xx} - T^{tx} v - p \implies \hat{\Pi} = \exp\Big[ \frac{\tau_{\Pi}}{\zeta} (T^{xx} - T^{tx} v - p) \Big]
\end{equation}
so our ODE can be written
\begin{equation}
\begin{aligned}
v' &= \frac{(-T^{xx} + T^{tx} v + p) \Big( 1 - \lambda (-T^{xx} + T^{tx} v + p) \Big)}{W v \Big[ \tau_{\Pi} T^{tx} ( p_{,e} v^{-2} - 1) + \tau_{\Pi} p_{,\rho} J^x W v^{-2} + \zeta W^2 v^{-1} \Big]}, \\
\end{aligned}
\end{equation}
where the RHS is entirely a function of $v$ after using the EOS to express $p = p(v)$.

\subsubsection{Rewriting the ODE in clearer form}

Recall that the bulk viscous characteristic speed is \cite{Bemfica19}
\begin{equation}
\mathfrak{c}^2 = \frac{\zeta}{\tau_{\Pi} (e + p + \Pi)} + p_{,e} + \frac{\rho \, p_{,\rho}}{e + p + \Pi}.
\end{equation}
With this in mind, we can rewrite the ODE for $v$ as
\begin{equation}
v' = \mathcal{N} \Big[ \tau_{\Pi} T^{tx} ( p_{,e} v^{-2} - 1) + \tau_{\Pi} p_{,\rho} J^x W v^{-2} + \zeta W^2 v^{-1} \Big]^{-1}
\end{equation}
where we have defined shorthand
\begin{equation}
\mathcal{N} \equiv \frac{(-T^{xx} + T^{tx} v + p) \Big( 1 - \lambda (-T^{xx} + T^{tx} v + p) \Big)}{W v}.
\end{equation}
We can now manipulate the equation for $v'$:
\begin{equation}
\begin{aligned}
v' &= \mathcal{N} \tau_{\Pi}^{-1} \Big[ T^{tx} ( p_{,e} v^{-2} - 1) + p_{,\rho} J^x W v^{-2} + \frac{\zeta}{\tau_{\Pi}} W^2 v^{-1} \Big]^{-1} \\
&= \mathcal{N} \tau_{\Pi}^{-1} \Big[ (e + p + \Pi) W^2 v ( p_{,e} v^{-2} - 1) + p_{,\rho} (\rho W v) W v^{-2} + \frac{\zeta}{\tau_{\Pi}} W^2 v^{-1} \Big]^{-1} \\
&= \mathcal{N} \tau_{\Pi}^{-1} W^{-2} v \Big[ (e + p + \Pi) v^2 ( p_{,e} v^{-2} - 1) + p_{,\rho} \rho + \frac{\zeta}{\tau_{\Pi}} \Big]^{-1} \\
&= \mathcal{N} \tau_{\Pi}^{-1} W^{-2} v (e + p + \Pi)^{-1} \Big[ p_{,e} + \frac{p_{,\rho} \rho}{e + p + \Pi} + \frac{\zeta}{\tau_{\Pi} (e + p + \Pi)} - v^2 \Big]^{-1} \\
&= \mathcal{N} \tau_{\Pi}^{-1} W^{-2} v (e + p + \Pi)^{-1} \Big[ \mathfrak{c}^2 - v^2 \Big]^{-1} \\
\end{aligned}
\end{equation}
which we can finally rewrite as
\begin{equation}
v' = \frac{v (-T^{xx} + T^{tx} v + p) \Big( 1 - \lambda (-T^{xx} + T^{tx} v + p) \Big)}{W \tau_{\Pi} T^{tx} [ \mathfrak{c}^2 - v^2 ]} \\
\end{equation}
where we have exchanged the $(e + p + \Pi)^{-1}$ factor for $T^{tx}$ in the denominator.

Written in this form, it is clear that the system fails to possess steady-state shockwave solutions (the denominator vanishes) if we ever have $v^2 = \mathfrak{c}^2$, in other words when the flow velocity crosses the local bulk viscous characteristic speed.

\textbf{TODO: check this equation in Mathematica}

\section{Deriving the $\hat{\Pi}$ evolution equation} \label{sec:Pi_hat_deriv}

Substituting (\ref{eq:Pi_E}) into the $\Pi$ evolution equation (\ref{eq:Pi_cov_EOM}) and rearranging yields:
\begin{equation}
\begin{aligned}
u^a \nabla_a \Pi &= \frac{1}{\tau_{\Pi}} (- \zeta \nabla_a u^a - \Pi) - \frac{\lambda}{\tau_{\Pi}} \Pi^2 \\
u^a \nabla_a \Pi + \frac{\zeta}{\tau_{\Pi}} \nabla_a u^a &= - \frac{1}{\tau_{\Pi}} \Pi ( 1 + \lambda \Pi ). \\
\end{aligned}
\end{equation}
We now define
\begin{equation} \label{eq:Pi_hat_defn}
\Pi = \frac{\zeta}{\tau_\Pi} \ln(\hat{\Pi}),
\end{equation}
in terms of which the above equation becomes
\begin{equation}
\begin{aligned}
u^a \nabla_a \Big[ \frac{\zeta}{\tau_\Pi} \ln(\hat{\Pi}) \Big] + \frac{\zeta}{\tau_{\Pi}} \nabla_a u^a &= - \frac{1}{\tau_{\Pi}} \Pi ( 1 + \lambda \Pi ) \\
\frac{\zeta}{\tau_\Pi} u^a \nabla_a \ln(\hat{\Pi}) + \ln(\hat{\Pi}) u^a \nabla_a \Big[ \frac{\zeta}{\tau_\Pi} \Big] + \frac{\zeta}{\tau_{\Pi}} \nabla_a u^a &= - \frac{1}{\tau_{\Pi}} \Pi ( 1 + \lambda \Pi ) \\
\frac{\zeta}{\tau_\Pi} u^a \nabla_a \ln(\hat{\Pi}) + \frac{\zeta}{\tau_{\Pi}} \nabla_a u^a &= - \frac{1}{\tau_{\Pi}} \Pi ( 1 + \lambda \Pi ) - \ln(\hat{\Pi}) u^a \nabla_a \Big[ \frac{\zeta}{\tau_\Pi} \Big] \\
\frac{\zeta}{\tau_\Pi} \frac{1}{\hat{\Pi}} u^a \nabla_a \hat{\Pi} + \frac{\zeta}{\tau_{\Pi}} \nabla_a u^a &= - \frac{1}{\tau_{\Pi}} \Pi ( 1 + \lambda \Pi ) - \ln(\hat{\Pi}) u^a \nabla_a \Big[ \frac{\zeta}{\tau_\Pi} \Big]\\
u^a \nabla_a \hat{\Pi} + \hat{\Pi} \nabla_a u^a &= - \frac{\hat{\Pi}}{\zeta} \Pi ( 1 + \lambda \Pi ) -  \hat{\Pi} \ln(\hat{\Pi}) u^a \nabla_a \ln \Big( \frac{\zeta}{\tau_\Pi} \Big) \\
\nabla_a (\hat{\Pi} u^a) &= - \frac{\hat{\Pi}}{\zeta} \Pi ( 1 + \lambda \Pi ) -  \hat{\Pi} \ln(\hat{\Pi}) u^a \nabla_a \ln \Big( \frac{\zeta}{\tau_\Pi} \Big). \\
\end{aligned}
\end{equation}
Expressing the right-hand side completely in terms of $\hat{\Pi}$ yields
\begin{equation} \label{eq:Pi_hat_eqn}
\nabla_a (\hat{\Pi} u^a) = - \hat{\Pi} \ln(\hat{\Pi}) \Big[ \frac{1}{\tau_\Pi} \Big( 1 + \frac{\zeta \lambda}{\tau_\Pi} \ln(\hat{\Pi}) \Big) + u^a \nabla_a \ln \Big( \frac{\zeta}{\tau_\Pi} \Big) \Big].
\end{equation}
Notice that in the case where $\zeta/\tau_\Pi$ is constant, the above equation is a conservation law.

\section{Entropy equation}

It may be useful to re-express the energy equation as an evolution equation for the entropy density.
We will need a couple of equations from thermodynamics, namely
\begin{align}
d e &= T ds + \mu \, d n \implies \frac{\partial e}{\partial s} \Big|_n = T, ~~~ \frac{\partial e}{\partial n} \Big|_s = \mu \label{eq:first_law} \\
e &= T s - p + \mu n \label{eq:Euler_relation}
\end{align}
which are the first law of thermodynamics and the Euler relation respectively, both in intrinsic form.
Going back to the energy equation, we have:
\begin{equation}
\begin{aligned}
0 &= u_b \nabla_{a} T^{ab} \\
&= u_b \nabla_{a} \Big[ e(s, n) u^a u^b + \tilde{p}(s, n, \Pi) \Delta^{ab} \Big] \\
&= u_b u^a u^b \nabla_{a} e + u_b e u^b \nabla_{a} u^a + u_b e u^a \nabla_{a} u^b + u_b \Delta^{ab} \nabla_{a} \tilde{p} + u_b \tilde{p} \nabla_{a} \Delta^{ab} \\
&= - u^a \nabla_{a} e - e \nabla_{a} u^a + u_b \tilde{p} \nabla_{a} (g^{ab} + u^a u^b) \\
&= - u^a \nabla_{a} e - e \nabla_{a} u^a - \tilde{p} \nabla_{a} u^a \\
&= u^a \nabla_{a} e  + (e + \tilde{p}) \nabla_{a} u^a \\
&= u^a \Big[ \frac{\partial e}{\partial s} \Big|_n \nabla_{a} s + \frac{\partial e}{\partial n} \Big|_s \nabla_a n \Big]  + (e + \tilde{p}) \nabla_{a} u^a \\
&= T u^a \nabla_{a} s + \mu u^a \nabla_a n  + (e + \tilde{p}) \nabla_{a} u^a \\
&= T \big[ \nabla_a (s u^a) - s \nabla_a u^a \big] + \mu \big[ \nabla_a (n u^a) - n \nabla_a u^a \big]  + (e + \tilde{p}) \nabla_{a} u^a \\
&= T \nabla_a (s u^a) + (e + \tilde{p} - T s - \mu n) \nabla_{a} u^a \\
&= T \nabla_a (s u^a) + (e + p - T s - \mu n) \nabla_{a} u^a + \Pi \nabla_a u^a \\
&= T \nabla_a (s u^a) + \Pi \nabla_a u^a, \\
\end{aligned}
\end{equation}
so rearranging slightly yields
\begin{equation} \label{eq:entropy_eqn}
\boxed{ \nabla_a (s u^a) = - \frac{\Pi}{T} \nabla_a u^a. }
\end{equation}

\section{Re-expressing $\mathcal{W}$ from the $\Pi$ equation} \label{sec:W_term}

The $\hat{\Pi}$ equation (\ref{eq:Pi_hat_eqn}) has an extra term, $\mathcal{W}$, which keeps it from being a conservation law.
We can manipulate that term a bit to express it as a coefficient times the divergence of the flow velocity; to do so, we first define
\begin{equation}
\xi(s, \rho) \equiv \frac{\zeta}{\tau_{\Pi}}
\end{equation}
which we have taken $\xi$ (WLOG) to be a function of the entropy density $s$ and the particle density $\rho$.
In terms of $\xi$, $\mathcal{W}$ can be written
\begin{equation}
\begin{aligned}
\mathcal{W} &\equiv u^a \nabla_a \ln \big[ \xi(s, \rho) \big] \\
&= u^a \Big[ \frac{\partial \ln(\xi)}{\partial \xi} \frac{\partial \xi}{\partial s} \Big|_{\rho} \nabla_a s + \frac{\partial \ln(\xi)}{\partial \xi} \frac{\partial \xi}{\partial \rho} \Big|_{s} \nabla_a \rho \Big] \\
&= \xi^{-1} \frac{\partial \xi}{\partial s} \Big|_{\rho} u^a \nabla_a s + \xi^{-1} \frac{\partial \xi}{\partial \rho} \Big|_{s} u^a \nabla_a \rho \\
&= \xi^{-1} \frac{\partial \xi}{\partial s} \Big|_{\rho} [\nabla_a (s u^a) - s \nabla_a u^a] + \xi^{-1} \frac{\partial \xi}{\partial \rho} \Big|_{s} [\nabla_a (\rho u^a) - \rho \nabla_a u^a] \\
&= \xi^{-1} \frac{\partial \xi}{\partial s} \Big|_{\rho} [- \frac{\Pi}{T} \nabla_a u^a - s \nabla_a u^a] - \xi^{-1} \frac{\partial \xi}{\partial \rho} \Big|_{s} \rho \nabla_a u^a \\
&= - \xi^{-1} \Big[ \frac{\partial \xi}{\partial s} \Big|_{\rho} \Big( \frac{\Pi}{T} + s \Big) + \frac{\partial \xi}{\partial \rho} \Big|_{s} \rho \Big] \nabla_a u^a, \\
\end{aligned}
\end{equation}
where we have used the entropy equation, (\ref{eq:entropy_eqn}), and the particle current conservation law, (\ref{eq:Ja_cons_law}), to arrive at the fifth line. 
Note that the result is now proportional to $\nabla_a u^a$.

\section{Identities from working out the $\Pi$ equation}

In working out the $\Pi$ equation we have used
\begin{equation}
\begin{aligned}
D_\Pi &= \mathcal{L}_\beta \Pi - \alpha v^a D_a \Pi \\
&= \beta^i \partial_i \Pi - \alpha v^i \partial_i \Pi \\
&= - \frac{u^i}{u^t} \partial_i \Pi \\
\end{aligned}
\end{equation}
and
\begin{equation}
\begin{aligned}
D_p &= - \frac{\zeta}{W \tau_{\Pi}} \Big[ \alpha v^a D_a W + \alpha W D_a v^a + W v^a D_a \alpha - \mathcal{L}_\beta W \Big] \\
&= - \frac{\zeta}{W \tau_{\Pi}} \Big[ D_a (\alpha W v^a) - \mathcal{L}_\beta W \Big] \\
&= - \frac{\zeta}{W \tau_{\Pi}} \Big[ \frac{1}{\sqrt{\gamma}} \partial_i (\sqrt{\gamma} \alpha W v^i) - \beta^i \partial_i W \Big]. \\
\end{aligned}
\end{equation}

\section{Transport coefficients \& constraints}

The system (\ref{eq:Tab_cons_law}-\ref{eq:Pi_E}) can be written as a first-order symmetric hyperbolic (FOSH) system, which immediately implies existence and uniqueness of solutions.
The equations are causal so long as the local characteristic speeds $\mathfrak{c} \leq 1$, where
\begin{equation}
\mathfrak{c}^2 = \frac{\zeta}{\tau_{\Pi} (e + \tilde{p})} + \frac{1}{\tilde{h}} \Big[ \frac{\tilde{p}}{\rho^2} \frac{\partial p}{\partial \epsilon} \Big|_\rho + \frac{\partial p}{\partial \rho} \Big|_\epsilon \Big];
\end{equation}
note the similarity between the second term and the characteristic speed (squared) for the relativistic Euler equations,
\begin{equation}
c_s^2 = \frac{1}{h} \Big[ \frac{p}{\rho^2} \frac{\partial p}{\partial \epsilon} \Big|_\rho + \frac{\partial p}{\partial \rho} \Big|_\epsilon \Big],
\end{equation}
which is also known as the sound speed (squared).
Written in this form, it is clear that we have $\mathfrak{c}^2 \to c_s^2$ in the limit where bulk viscosity vanishes, $\zeta/\tau_{\Pi}, \Pi \to 0 \implies \tilde{p} \to p, \tilde{h} \to h$.

The authors of \cite{Bemfica19} also state that in the ``near equilibrium'' regime $\Pi/(e+p) \ll 1$, solutions to the equations of motion are linearly stable about thermodynamic equilibrium and are consistent with the second law of thermodynamics (divergence of the entropy current is non-negative).

The quantities $\zeta, \tau_{\Pi}, \lambda$ are known as \textit{transport coefficients} and are functions of $\epsilon, \rho$; $\zeta$ is the bulk viscous coefficient, $\tau_{\Pi}$ is a relaxation time that controls how quickly bulk viscosity acts on the solution, and $\lambda$ is the magnitude of a ``higher order'' term which was included in \cite{Bemfica19} because it did not impact the proofs of the statements mentioned in the previous paragraph.

\section{Rewriting the characteristic speed}

Bemfica et al. \cite{Bemfica19} take the pressure to be $p(e,n)$ rather than our choice of variables $p(\epsilon, \rho)$.
For this reason, they give the sound speed to be
\begin{equation} \label{eq:BDN_csSq}
c_s^2 = \frac{\partial p}{\partial e} \Big|_n + \frac{n}{e+p} \frac{\partial p}{\partial n} \Big|_e.
\end{equation}

The thermodynamic variables are related by
\begin{align}
\rho &= m n \\
e    &= \rho (1 + \epsilon),
\end{align}
where $m$ is the particle mass and is taken to be constant.
Taking a total derivative of the second equation yields
\begin{equation}
\begin{aligned}
de &= (1 + \epsilon) d \rho + \rho d \epsilon \\
\implies \frac{\partial \epsilon}{\partial e} \Big|_\rho &= \frac{1}{\rho} \\
\implies \frac{\partial \epsilon}{\partial \rho} \Big|_e &= -\frac{(1+\epsilon)}{\rho} = - \frac{e}{\rho^2} \\
\implies \frac{\partial \rho}{\partial e} \Big|_\rho &= 0. \\
\end{aligned}
\end{equation}
So we can now express the derivatives in (\ref{eq:BDN_csSq}) as
\begin{equation} \label{eq:dpde_n}
\begin{aligned}
\frac{\partial p}{\partial e} \Big|_n &= \frac{\partial p(\epsilon, \rho)}{\partial e} \Big|_\rho \\
&= \frac{\partial p}{\partial \epsilon} \Big|_\rho \frac{\partial \epsilon}{\partial e} \Big|_\rho + \frac{\partial p}{\partial \rho} \Big|_\epsilon \frac{\partial \rho}{\partial e} \Big|_\rho \\
&= \frac{\partial p}{\partial \epsilon} \Big|_\rho \cdot \frac{1}{\rho} + \frac{\partial p}{\partial \rho} \Big|_\epsilon \cdot 0 \\
&= \frac{1}{\rho} \frac{\partial p}{\partial \epsilon} \Big|_\rho \\
\end{aligned}
\end{equation}
and
\begin{equation} \label{eq:dpdn_e}
\begin{aligned}
\frac{\partial p}{\partial n} \Big|_e &= m \frac{\partial p(\epsilon, \rho)}{\partial \rho} \Big|_e \\
&= m \frac{\partial p}{\partial \epsilon} \Big|_\rho \frac{\partial \epsilon}{\partial \rho} \Big|_e + m \frac{\partial p}{\partial \rho} \Big|_\epsilon \frac{\partial \rho}{\partial \rho} \Big|_e \\
&= m \frac{\partial p}{\partial \epsilon} \Big|_\rho \Big( -\frac{e}{\rho^2} \Big) + m \frac{\partial p}{\partial \rho} \Big|_\epsilon \\
&= -\frac{m e}{\rho^2} \frac{\partial p}{\partial \epsilon} \Big|_\rho + m \frac{\partial p}{\partial \rho} \Big|_\epsilon. \\
\end{aligned}
\end{equation}

Inserting these pressure derivatives into the sound speed (\ref{eq:BDN_csSq}) yields
\begin{equation} \label{eq:csSq_eps_rho}
\begin{aligned}
c_s^2 &= \frac{1}{\rho} \frac{\partial p}{\partial \epsilon} \Big|_\rho + \frac{n}{e+p} \Big[ -\frac{m e}{\rho^2} \frac{\partial p}{\partial \epsilon} \Big|_\rho + m \frac{\partial p}{\partial \rho} \Big|_\epsilon \Big] \\
&= \frac{1}{\rho} \frac{\partial p}{\partial \epsilon} \Big|_\rho - \frac{e}{\rho (e+p)} \frac{\partial p}{\partial \epsilon} \Big|_\rho + \frac{\rho}{e+p} \frac{\partial p}{\partial \rho} \Big|_\epsilon \\
&= \frac{p}{\rho^2 h} \frac{\partial p}{\partial \epsilon} \Big|_\rho + \frac{1}{h} \frac{\partial p}{\partial \rho} \Big|_\epsilon \\
&= \frac{1}{h} \Big[ \frac{p}{\rho^2} \frac{\partial p}{\partial \epsilon} \Big|_\rho + \frac{\partial p}{\partial \rho} \Big|_\epsilon \Big] \\
\end{aligned}
\end{equation}
where we have used the definition of the specific enthalpy, $h \equiv (e+p)/\rho$, and this result agrees with \cite{RezzollaZanotti,Deppe22}.

We can now use the pressure derivatives (\ref{eq:dpde_n}-\ref{eq:dpdn_e}) to rewrite the bulk-only MIS system's characteristic speed \cite{Bemfica19}:
\begin{equation}
\begin{aligned}
\mathfrak{c}^2 &= \frac{\zeta}{\tau_{\Pi} (e + p + \Pi)} + \Big( \frac{\partial p}{\partial e} \Big)_n + \frac{n}{e+p+\Pi} \Big( \frac{\partial p}{\partial n} \Big)_e \\
&= \frac{\zeta}{\tau_{\Pi} (e + p + \Pi)} + \frac{1}{\rho} \frac{\partial p}{\partial \epsilon} \Big|_\rho + \frac{n}{e+p+\Pi} \Big( -\frac{m e}{\rho^2} \frac{\partial p}{\partial \epsilon} \Big|_\rho + m \frac{\partial p}{\partial \rho} \Big|_\epsilon \Big) \\
&= \frac{\zeta}{\tau_{\Pi} (e + p + \Pi)} + \Big[ \frac{1}{\rho} - \frac{e}{\rho (e+p+\Pi)} \Big] \frac{\partial p}{\partial \epsilon} \Big|_\rho + \frac{\rho}{e+p+\Pi} \frac{\partial p}{\partial \rho} \Big|_\epsilon \\
&= \frac{\zeta}{\tau_{\Pi} (e + p + \Pi)} + \Big[ \frac{p + \Pi}{\rho (e+p+\Pi)} \Big] \frac{\partial p}{\partial \epsilon} \Big|_\rho + \frac{\rho}{e+p+\Pi} \frac{\partial p}{\partial \rho} \Big|_\epsilon \\
&= \frac{\zeta}{\tau_{\Pi} (e + p + \Pi)} + \frac{\rho}{e+p+\Pi} \Big[ \frac{p + \Pi}{\rho^2} \frac{\partial p}{\partial \epsilon} \Big|_\rho + \frac{\partial p}{\partial \rho} \Big|_\epsilon \Big] \\
&= \frac{\zeta}{\tau_{\Pi} (e + \tilde{p})} + \frac{1}{\tilde{h}} \Big[ \frac{\tilde{p}}{\rho^2} \frac{\partial p}{\partial \epsilon} \Big|_\rho + \frac{\partial p}{\partial \rho} \Big|_\epsilon \Big] \\
\end{aligned}
\end{equation}
where we have defined
\begin{equation}
\tilde{p} = p + \Pi, ~~~ \tilde{h} = \frac{e + \tilde{p}}{\rho}
\end{equation}
Note the similarity between the second term in the last line for $\mathfrak{c}^2$ and the relativistic Euler sound speed (\ref{eq:csSq_eps_rho}).

\end{document}
